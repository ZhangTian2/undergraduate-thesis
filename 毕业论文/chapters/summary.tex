% !TeX root = ../main.tex
\chapter{总结}
\paragraph*{}
本文着眼于探测金球与金板之间的Casimir效应,并利用金球与金板之间的Casimir力研究制备无接触式扫描显微镜。主要分为以下四个部分。
\paragraph*{}
第一部分介绍了Casimir效应的研究背景以及Casimir力与van der Waals力之间的关系。1948年Casimir研究分子间相互作用时在Bohn的提示下将研究对象从具体的分子转为场,提出了由于真空零点能的存在,两个不带电平行理想导体板之间存在吸引力。随后Lifshitz研究了介质材料的介电函数和磁导率对Casimir效应的影响。Derjaguin等人研究了物体形状对于Casimir效应的影响。我们实验中的理论力曲线就是参考了Lifshitz和
Derjaguin的理论公式。关于Casimir效应实验上的研究源于1958年,现代研究Casimir效应的装置主要是仿照Mohideen组,使用原子力显微镜探测材料之间的Casimir力。
\paragraph*{}
第二部分介绍了研究Casimir效应的理论方法。包括半经典理论,维度正则化理论,世界线理论和Lifshitz理论。前三个理论计算的都是理想导体板之间的Casimir效应,他们的起始点都不同,但是得出的结果却是相同的,半经典理论认为真空零点能对应的电磁场在导体板之间以驻波的形式存在,并引入一个加权函数来避免能量计算结果的发散;维度正则化理论利用Riemann zeta函数的特殊性质直接计算出有限的Casimir能;而世界线理论从完全不同的角度出发,更方便于计算机模拟。最后Lifshitz理论将理想导体板的情形推广到介质板,提供了验证实验结果和预测新实验现象的指导方法。
\paragraph*{}
第三部分介绍了研究Casimir效应的实验原理与实验设计。具体包括原子力显微镜的工作原理,样品及探针的制作步骤,实验前的准备操作和研究Casimir效应实验的三个阶段。这三个阶段分别为阶段一:利用仪器自带的程序测量Casimir力曲线。阶段二:利用外接端口读取四象限信号,并且控制扫描管的移动。阶段三:对台阶样品进行无接触式扫描,研究基于Casimir效应的无接触式扫描显微镜的可行性。之后又介绍了实验重点难点以及对应的解决措施。
\paragraph*{}
第四部分介绍了实验上对于Casimir效应的研究,包括测量金球与金板的力-距离曲线以及对金台阶样品的恒高无接触式扫描。利用仪器自带的程序测量得到的力-距离曲线和理论曲线相符。之后我们又尝试了用外接端口测量力-距离曲线,也和理论曲线相符。在此基础上,我们尝试了采集恒高模式下无接触扫描金台阶样品得到的力信号,发现探针与样品相距很近时能够感应到台阶的存在,但是随着距离的增大,台阶信号最终会淹没在噪声信号之中。
\paragraph*{}
总之,本文介绍了实验上对扫描Casimir力显微镜的研究,最终实现了利用针尖采集到的力信号远距离探测金台阶样品的功能。