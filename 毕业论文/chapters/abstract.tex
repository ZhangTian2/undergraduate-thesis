% !TeX root = ../main.tex

\ustcsetup{
  keywords = {
    Casimir效应;零点能;Lifshitz理论;无接触式扫描显微镜
  },
  keywords* = {
    Casimir effect; zero point energy; Lifshitz theory; non-contact scanning microscope  
  }
}

\begin{abstract}
Casimir效应是源于零点能的由电磁场涨落引起的物体之间的相互作用,是一种宏观量子现象。研究Casimir效应不仅对于量子力学理论的发展有指导意义,也有利于控制微纳器件之间的相互作用,解决工程问题。本文首先介绍了研究Casimir效应的背景和理论,然后利用原子力显微镜装置在实验上探测金球与金板之间的Casimir效应,最后利用金球与金板之间的Casimir效应研究制备无接触式扫描显微镜。论文主要包含以下四章内容:


第一章,主要介绍了Casimir效应的研究背景,Casimir力与van der Waals力之间的关系以及研究Casimir效应的理论方法。包括半经典理论,维度正则化理论,世界线理论和Lifshitz理论。


第二章,主要介绍了研究Casimir效应的实验原理与设计。包括原子力显微镜装置的工作原理,利用磁控溅射和电子束热蒸发装置制备探针和样品,实验前的准备操作以及实验的重难点。


第三章,主要介绍了实验上对于Casimir效应的研究,包括测量金球与金板的力-距离曲线以及对金台阶样品的恒高无接触式扫描。测量得到的力-距离曲线和理论拟合的很好。基于测量得到的力曲线,进一步研究了通过采集恒高模式下无接触扫描金台阶样品得到的力信号来得出样品表面形貌的可能性。


最后,在第四章中对本论文的内容进行了总结。


笔者编写的信号采集和处理的程序附在了附录中。
\end{abstract}

\begin{enabstract}
  Casimir effect, a macroscopic quantum phenomenon, is the interaction between objects caused by the fluctuation of electromagnetic field from the zero-point energy.  The study of Casimir effect is not only of guiding significance for the development of quantum mechanics theory, but also conducive to controlling the interaction between micro and nano devices and solving engineering problems.  In this paper, we firstly introduce the background and theory of Casimir effect, and then the Casimir effect between the gold sphere and the gold plate is detected experimentally by Atomic Force Microscope. Finally, by using the Casimir effect between the gold sphere and the gold plate,we study the possibility of inventing non-contact scanning microscope. The thesis mainly includes the following four chapters:  
 
 
  In the first chapter, the research background of Casimir effect, the relationship between Casimir force and van der Waals force and the theoretical method of studying Casimir effect, which include semi-classical theory, dimensional regularization theory, world line theory and Lifshitz theory are introduced.
   
   
  In the second chapter, the experimental principle and design of Casimir effect are introduced.  It includes the working principle of the AFM apparatus, the preparation of probes and samples by magnetron sputtering and electron beam thermal evaporation apparatus, the preparation operation before the experiment, and the important and difficult points of the experiment.  
   
   
  In the third chapter, we mainly introduce the experimental study of Casimir effect, including the measurement of force-distance curve between the gold sphere and the gold plate and the constant height non-contact scanning of the gold step sample.  The measured force-distance curve fits well with the theory. Based on the measured force curve, the possibility of obtaining the surface topography of the sample by collecting the force signal obtained from the non-contact scanning gold step samples under constant height mode was further studied.  
   
   
  Finally, in the fourth chapter, we summarize the content of this paper.   


  The program of signal acquisition and processing prepared by the author is attached in the appendix.
\end{enabstract}
